% \documentclass[dvipdfmx, 11pt]{beamer}
\documentclass[aspectratio=169, dvipdfmx, 11pt]{beamer} % aspectratio=43, 149, 169
\usepackage{here, amsmath, latexsym, amssymb, bm, ascmac, mathtools, multicol, tcolorbox, subfig}

%デザインの選択(省略可)
\usetheme{Luebeck}
%カラーテーマの選択(省略可)
\usecolortheme{orchid}
%フォントテーマの選択(省略可)
\usefonttheme{professionalfonts}
%フレーム内のテーマの選択(省略可)
\useinnertheme{circles}
%フレーム外側のテーマの選択(省略可)
\useoutertheme{infolines}
%しおりの文字化け解消
\usepackage{atbegshi}
\ifnum 42146=\euc"A4A2
\AtBeginShipoutFirst{\special{pdf:tounicode EUC-UCS2}}
\else
\AtBeginShipoutFirst{\special{pdf:tounicode 90ms-RKSJ-UCS2}}
\fi
%ナビゲーションバー非表示
\setbeamertemplate{navigation symbols}{}
%既定をゴシック体に
\renewcommand{\kanjifamilydefault}{\gtdefault}
%タイトル色
\setbeamercolor{title}{fg=structure, bg=}
%フレームタイトル色
\setbeamercolor{frametitle}{fg=structure, bg=}
%スライド番号のみ表示
%\setbeamertemplate{footline}[frame number]
%itemize
\setbeamertemplate{itemize item}{\small\raise0.5pt\hbox{$\bullet$}}
\setbeamertemplate{itemize subitem}{\tiny\raise1.5pt\hbox{$\blacktriangleright$}}
\setbeamertemplate{itemize subsubitem}{\tiny\raise1.5pt\hbox{$\bigstar$}}
% color
\newcommand{\red}[1]{\textcolor{red}{#1}}
\newcommand{\green}[1]{\textcolor{green!40!black}{#1}}
\newcommand{\blue}[1]{\textcolor{blue!80!black}{#1}}

\title[桁DP]{桁DP}
\subtitle{Digit DP}
\author{task4233}
\date{\today}

\begin{document}
\maketitle

\begin{frame}{目次}
    \tableofcontents
\end{frame}

\section{Section 1}
\begin{frame}{目次}
    \tableofcontents[currentsection]
\end{frame}

\begin{frame}{ブロック環境}
    \begin{block}{block}
    block
    \end{block}
    \begin{alertblock}{alertblock}
    alertblock
    \end{alertblock}
    \begin{exampleblock}{exampleblock}
    exampleblock
    \end{exampleblock}
    \begin{tcolorbox}[colframe=green,
    colback=green!10!white,
    colbacktitle=green!40!white,
    coltitle=black, fonttitle=\bfseries,
    title=My box]
        box contents
    \end{tcolorbox}
\end{frame}

\section{Section 2}
\begin{frame}{目次}
    \tableofcontents[currentsection]
\end{frame}

\begin{frame}{箇条書き}
    \begin{itemize}
    \item アイテム1
    \item \alert{アイテム2}
        \begin{itemize}
        \item アイテム1
        \item \alert{アイテム2}
            \begin{itemize}
            \item アイテム1
            \item \alert{アイテム2}
            \end{itemize}
        \end{itemize}
    \end{itemize}
    \[
    \bm{x}^\top\bm{y}
    \]
    \begin{enumerate}
    \item abcde
    \item \structure{ABCDE}
    \item 
    \end{enumerate}
\end{frame}

\end{document}